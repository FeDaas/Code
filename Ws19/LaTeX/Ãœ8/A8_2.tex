\documentclass{article}
\usepackage[russian,english,french,ngerman]{babel}
\usepackage[T1]{fontenc}
\usepackage{CJKutf8}
\usepackage{CJKulem}  
\begin{document}
	\selectlanguage{ngerman}
		Babel ermöglicht, das befehle auf anderen Sprachen ausgeführt werden. Zum Beispiel ist
		heute der \today \\
	\selectlanguage{french}
		Mais aujourd'hui en france c'est \today \\
		De plus, la césure babel, c.-à-d. si vous prenez un long mot, par exemple anticonstitutionnellement il est bien séparé \\
	\selectlanguage{russian}
		просто глупо, что вы не включили набор символов. Для "ä" \ и нелатинских символов вы должны использовать [T1] {fontenc}
	
	\selectlanguage{ngerman}
		Also ist es schon sinnvoll babel zu verwenden wenn man befehle benutzen will, die auch das layout beeinflussen. Wenn man nur eine andere Sprache benutzen will ohne auf die Konventionen der Sprache zu achten braucht man babel nicht  

	
	
\end{document}