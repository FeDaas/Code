% !TEX program = xelatex
% !TEX encoding = UTF-8 Unicode
% !TEX spellcheck = de_DE

\documentclass{scrreprt}

\usepackage{blindtext,booktabs,color,nicefrac,polyglossia,setspace,xltxtra,yfonts,hyperref,tabulary,setspace,makeidx,}

\makeindex

\setmainlanguage{german}

\setromanfont[Mapping=tex-text]{Linux Libertine O}
\setsansfont[Mapping=tex-text]{Linux Biolinum O}

\onehalfspacing




\begin{document}
	
\begin{titlepage}
	\vspace*{2cm}
	\large
	\begin{center}
		---	Dies ist ein Schmutztitel ---
	\end{center}
\end{titlepage}

\begin{titlepage}
	
	\begin{centering}
		\doublespacing
		\begin{bfseries}
			\Large 
			Ruprecht-Karls-Universität Heidelberg \\
			Fakultät für Mathematik und Informatik \\ 
			Institut für Informatik \\[3cm]
			Masterarbeit \\
			\Huge 
			Titel Deutsch \\
			Titel Englisch \\
		\end{bfseries}
	\end{centering}
	
	\begin{table}[b]
		\doublespacing
		\large
		\begin{tabular}{ll}
			Name: & Max Mustermann \\
			Matrikelnummer: & 0000000 \\
			BetreuerIn: & Martha Musterfrau \\
			Datum der Abgabe: & 01.01.1999
		\end{tabular}
	\end{table}
	
\end{titlepage}


\subsection*{Abstract}
Dieses Dokument dient der Übung des Satzes von umfangreichen Projekten\index{Projekten} in \LaTeX\index{LaTeX@\LaTeX}. Es gehört zum siebten Übungszettel des \LaTeX-Kurses im Wintersemester 2016\,/\,17. Inhaltlich hat es so ziemlich nichts zu bieten, es könnte aber interessant sein, sich den zugehörigen Sourcecode mal anzusehen, da er eine Menge interessanter \LaTeX-Kommandos enthält.


\tableofcontents


\setchapterpreamble[o]{\dictum[W. Busch]{Stets findet Überraschung statt. Da, wo man's nicht erwartet hat.}}

\chapter{Einleitung}

\blindtext$\sin(x)\cdot\cos(x) = -\nicefrac{1}{2} \cos(2x)$\footnote{Man beachte auch, dass $\sin(x\pm y) = \sin(x)\cos(y) \pm \cos(x)\sin(y)$}

\vfill

\begin{table}[h]
  \centering
  \begin{tabular}{ccc}
    \toprule
    a & b & c\\
    d & e & f\\
    g & h & i\\
    \bottomrule
  \end{tabular}
  \caption{Die erste Tabelle}
  \label{table - first table}
\end{table}

\setchapterpreamble[o]{\dictum[Cato]{\textsc{Nullus est liber tam malus, ut non aliqua parte prosit.}}}

\index{Blinddokument|(}
\blinddocument
\index{Blinddokument|)}


\begin{figure}[b]
  \centering
  \fbox{I am a picture!}
  \caption{Ein Bild, das die Aussage des Textes unterstreicht.}
  \label{picture - statement}
\end{figure}






\setchapterpreamble[o]{\dictum[U.\,R. Heber]{Ein schlauer Spruch bereichert den Kapitel-Anfang.}}

\chapter{Ein weiteres Kapitel}

\begin{figure}[h]
  \centering
  \fbox{I am a picture, introducing this chapter!}
  \caption{Bildunterschrift}
  \label{picture - introduction}
\end{figure}

\index{Blindtext}
\Blindtext\footnote{Man beachte auch, dass $\sin(x\pm y) = \sin(x)\cos(y) \pm \cos(x)\sin(y)$}

\vfill

\begin{table}[h]
  \centering
  \begin{tabular}{ccc}
  \toprule
  eins & zwei & drei\\
  vier & fünf & sechs\\
  sieben & acht & neun\\
  \bottomrule
  \end{tabular}
  \caption{Eine Tabelle mit neun Einträgen}
  \label{table - nine entries}
\end{table}

\setchapterpreamble[o]{\dictum[F. Halm]{\hspace*{2em}\textfrak{Ruhe bleibt den Leichen;\\ Der Lebende tauch' frisch ins: Lebens:meer.}}}


\chapter{Und noch ein weiteres Kapitel}

\begin{figure}[h]
  \centering
  \fbox{I am a picture!}
  \caption{Beispiel zu diesem Kapitel}
  \label{picture - example}
\end{figure}

\index{Blindtext}
\Blindtext Der \verb$\Blindtext$-Befehl\index{Blindtext} ist eine nette Sache\index{Sache}, wenn man in \LaTeX\index{LaTeX@\LaTeX} sehen will, wie ein Dokument mit viel Inhalt aussieht, ohne dass man diesen hat.

\begin{figure}[p]
  \centering
  \fbox{I am a picture!}
  \caption{Veranschaulichung der Aussage}
  \label{picture - illustration}
\end{figure}

\begin{figure}[p]
  \centering
  \fbox{I am a picture!}
  \caption{Detailansicht}
  \label{picture - detail}
\end{figure}

\begin{figure}[p]
  \centering
  \fbox{I am a picture!}
  \caption{Visualisierung des Ergebnisses}
  \label{picture - visualization}
\end{figure}



\appendix 
\listoffigures 
\listoftables

\printindex


\end{document}
